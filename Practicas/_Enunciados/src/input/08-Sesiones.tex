% !TEX root = ../Sesiones-TDA-Ejercicios.tex




\end{document}


\etocsetnexttocdepth{3}
\etocsettocstyle{\hrule \vskip 0.15cm \subsubsection*{Índice Parcial}\vskip 0cm}{\vskip 0.15cm\hrule}
\localtableofcontents

\

\

\textbf{Metodología General:}

\begin{itemize}
\item El profesor explicará los fundamentos de los TDAs sobre los que se trabajará en cada sesión, si fuera necesario.
\item Los miembros del grupo se repartirán las tareas a realizar.
\item Cada uno desarrollará la parte que le haya sido asignada.
\item Finalizada la parte asignada, le explicará al otro miembro del grupo todo lo que ha realizado con la máxima exactitud posible, justificando las decisiones adoptadas para llegar a la solución del ejercicio.
\item En la medida de lo posible, los miembros del grupo no deberían trabajar sobre la misma tarea salvo que alguno de los miembros no sepa cómo seguir y necesitara ayuda para continuar.
\item Ello no significa que cada uno trabaje por su cuenta pues cada miembro del grupo siempre debe rendir cuentas del trabajo que esté realizando.
\end{itemize}



\newpage


%%%%%%%%%%%%%%%%%%%%%%%%%%%%%%%%%%%%%%%%%%%%%%%%%%%%%%%%%%%%%%%%%%%%%%%%%%%%%%%%%%%%%%%%%%%%%%%%%%%%%%%%%%%%%%%%
%%%%%%%%%%%%%%%%%%%%%%%%%%%%%%%%%%%%%%%%%%%%%%%%%%%%%%%%%%%%%%%%%%%%%%%%%%%%%%%%%%%%%%%%%%%%%%%%%%%%%%%%%%%%%%%%
\section*{Sesión Práctica 1 }
\addcontentsline{toc}{section}{Sesión Práctica 1} 

\begin{enumerate}% [label=\sc{Paso} \arabic*.]

\item 
El profesor te explicará todo lo referente a la programación estructurada en Python (pág. \pageref{subsec:ProgramacionEstructurada}), así como la construcción de funciones (pág. \pageref{subsec:ProgramacionProcedimental}) (en su versión más simple) y construcción de clases en Python (pág. \pageref{subsec:ClasesEnPython}), para que lo compares con Processing y empezar a programar en esta misma sesión. 

\item 
El profesor te explicará todo lo referente a la creación de un nuevo proyecto en PyCharm (pág. \pageref{subsec:SesionPyCharm}) para crear un proyecto llamado \texttt{SesionesPracticas}. Presta especial atención al significado de entorno virtual. 

\item Construye en el proyecto la  estructura \directory{SesionesPracticas/SesionP1} y desarrolla todo lo que viene a continuación en \directory{SesionP1}.


\item Crearéis el fichero \directory{SesionP1/EjemDepuracion.py} con el siguiente contenido para aprender a ejecutar y a depurar un fichero Python en PyCharm.

\begin{pyverbatim}[][frame=single, numbers=left, numbersep=2pt]
def es_par(numero: int) -> bool:
    if numero % 2 == 0:
        return True
    else:
        return False


def suma_pares(lista_numeros: list) -> int:
    suma: int = 0
    pos: int = 0
    while pos < len(lista_numeros):
        if es_par(lista_numeros[pos]):
            suma += lista_numeros[pos]
        pos += 1
    return suma


numeros: list = [1, 2, 3, 4, 5, 6, 7, 8, 9, 10]
resultado_suma_pares: int = suma_pares(numeros)
print(f"La suma de los números pares en la lista es {resultado_suma_pares}")

\end{pyverbatim}

Aprende bien lo siguiente: crear y borrar puntos de corte, \textit{Resume Program}, \textit{Step Over}, \textit{Step Into My Code}, \textit{Step Out} y evaluar expresiones. Para ello sigue los siguientes pasos:

\begin{enumerate}[label=\arabic*)]
\item Marca como breakpoints la línea 12 y selecciona 'Debug EjemDepuracion'. Se mostrará la ventana 'Threads \& Variables'.
\item ¿Qué variables se muestran y cuáles son sus valores? Despliega los valores de {\tt lista\_numeros}.
\item ¿Cuanto vale {\tt len(lista\_numeros)}? Usa el campo 'Evaluate expression' y verás el resultado. 
\item Añade {\tt len(lista\_numeros)} al resto de variables pulsando el símbolo '+' que está a la derecha del campo 'Evaluate expressión'.
\item Usa \textit{Step Over}: \textit{Pasa por encima} de la línea actual de código y te lleva a la siguiente línea incluso si la línea resaltada tiene llamadas a métodos o funciones. Observa que saltas a la línea 14.
\item Usa \textit{Resume Program}: Te permite continuar el programa hasta llegar a un nuevo punto de corte. Observa que ejecutas el resto del bucle y te vuelves a parar en la línea 12.
\item Pon un punto de corte en la línea 2. Y vuelve a ejecutar \textit{Resume Program}. Observa que ahora no te paras en la línea 12, sino en la línea 2. ¿Cuánto vale ahora {\tt len(lista\_numeros)}?
\item Quita el punto de corte de la línea 12.
\item Usa \textit{Step Out}: Saldrás de la función actual y le llevará a línea donde se invocó a la función. Observa que se para en la línea 12. Lo entenderás mejora si usa \textit{Resume Program} y después \textit{Step Out}, una y otra vez. Siempre se parará en la línea 12, pero sin tener un punto de corte.
\item Usa \textit{Step Into}: Funciona igual que \textit{Step Over}, te pasará a la siguiente línea pero ahora el \textit{paso se adentrará} en el método para mostrar lo que sucede en su interior. Utilice esta opción cuando no esté seguro de que el método devuelve un resultado correcto. Pulsa muchas veces esta opción y recorre el bucle while un par de veces (al menos).
\item \textit{Step Into my Code} es similar a \textit{Step Into} pero con la diferencia de que te permirte concentrarse en tu propio código y evitar que el depurador entre en las clases de las bibliotecas.
\end{enumerate}



Recuerda, la \underline{depuración} es una destreza fundamental en programación. Cuando hagas un programa, \textbf{no preguntes al profesor por qué no funciona si antes no lo has depurado} (es decir, haber comprobado el flujo de control y los valores que adoptan las variables)



\item 
El profesor te recordará qué es una especificación informal, la función de abstracción y el invariante de la representación. También se te explicará qué es un método mágico y cuándo son invocados. Verás cómo se aplican todos estos conceptos en Python desarrollando el Ejercicio \ref{sec:numerosPositivos} junto con el profesor, para ello crearéis la solución en el fichero \directory{SesionP1/Positivo.py}. 

Recuerda:
\begin{itemize}
\item Especificación Informal: Es documentar el código (sección \ref{sec:documentarCodigo}).
\item Función de abstracción: Usar el TDA adecuado (página \pageref{sec:representacionTDA})
\item Invariante de la representación: Usar assert/raise (página \pageref{sec:representacionTDA}).
\end{itemize}


\item Desarrolla el Ejercicio \ref{sec:FechaSimplePositivos} con las consideraciones del Ejercicio \ref{sed:TDAFechaSimple}

\item Para seguir con los objetivos de comprender qué es un TDA, su especificación informal, su función de abstracción, su invariante de la representación y su implementación, desarrolla los ejercicios \ref{sec:TDACilindro} y \ref{sec:TDAFraccion}.
\end{enumerate}



\


\



%%%%%%%%%%%%%%%%%%%%%%%%%%%%%%%%%%%%%%%%%%%%%%%%%%%%%%%%%%%%%%%%%%%%%%%%%%%%%%%%%%%%%%%%%%%%%%%%%%%%%%%%%%%%%%%%
%%%%%%%%%%%%%%%%%%%%%%%%%%%%%%%%%%%%%%%%%%%%%%%%%%%%%%%%%%%%%%%%%%%%%%%%%%%%%%%%%%%%%%%%%%%%%%%%%%%%%%%%%%%%%%%%
\section*{Sesión Práctica 2 }
\addcontentsline{toc}{section}{Sesión Práctica 2} 

\begin{enumerate}% [label=\sc{Paso} \arabic*.]

\item El profesor te hará una exposición rápida entre las colecciones secuenciales y no secuenciales, distinguiendo cuáles son mutables y cuáles no. Para cada una de ellas te indicará qué representan, sus constructores y operadores habituales, así como como el uso del \cm{for} para realizar iteraciones sobre estas colecciones.
Recuerda que:

\begin{itemize}
\item Memoriza los distintos tipos de contenedores de Python e intenta memorizar la mayor cantidad de métodos y operaciones que se pueden realizar con cada uno.

\item \cm{list(iterable)}, \cm{tuple(iterable)}, \cm{set(iterable)}, \cm{dict(iterable)} realmente no son constructores como tales, sino que se usan como funciones de conversión de forma similar a \cm{int(string)} o \cm{str(numero)}. 

\item Un colección debe inicializarse como una colección vacía y, poco a poco, vamos añadiendo elementos.
\end{itemize}

\item Muchas colecciones se crean por comprehensión. El profesor te explicará con detenimiento la Construcción de Contenedores por Comprehensión (página \pageref{sec:Comprehension}) y desarrollaréis el Ejercicio \ref{sec:ejercComprehension}.

\item Para trabajar con listas y diccionarios de Python, desarrolla el Ejercicio \ref{sec:estadistica}.


\item Utilizando las listas de Python, implementa el TDA Bag. Ejercicio \ref{sec:TDABag}. Llama al fichero \texttt{BagList.py}.

\item Para trabajar con listas, diccionarios y bags, implementa el Ejercicio \ref{sec:anagramas}. Lee antes cómo hacer iteraciones en las colecciones de Python (pág. \pageref{sec:iteracionColecciones}).

\item Para trabajar con conjuntos y diccionarios implementa el Ejercicio \ref{sec:cursosDeTitulo}.
\end{enumerate}


\


\


%%%%%%%%%%%%%%%%%%%%%%%%%%%%%%%%%%%%%%%%%%%%%%%%%%%%%%%%%%%%%%%%%%%%%%%%%%%%%%%%%%%%%%%%%%%%%%%%%%%%%%%%%%%%%%%%
%%%%%%%%%%%%%%%%%%%%%%%%%%%%%%%%%%%%%%%%%%%%%%%%%%%%%%%%%%%%%%%%%%%%%%%%%%%%%%%%%%%%%%%%%%%%%%%%%%%%%%%%%%%%%%%%
\section*{Sesión Práctica 3}
\addcontentsline{toc}{section}{Sesión Práctica 3} 

\begin{enumerate}% [label=\sc{Paso} \arabic*.]

\item El profesor te explicará qué es un iterador y un iterable, y cómo se implementa en Python.  Como ejemplo, desarrollaréis el Ejercicio \ref{sec:iteradorListas}.

\item El profesor también te explicará como convertir cualquier colección en una colección `indexable' usando los métodos mágicos \cm{\_\_getitem\_\_()} y \cm{\_\_setitem\_\_()}. En este sentido, te recordará qué son los métodos mágicos y te explicará  los fundamentos del ejercicio \ref{sec:datosIndexados}.


\item Implementa el tipo de datos array 1D y 2D: ejercicios \ref{sec:arrayUnoD}, \ref{sec:arrayDosD}, 


\item Implementa los métodos mágicos que te permitan acceder a los datos de forma indexada. Ejercicio \ref{sec:datosIndexados}.

\item Por último, adapta el Array2D para trabajar con matrices: Ejercicio \ref{sec:MatrizDatosIndexados}. 

\end{enumerate}




\


\


%%%%%%%%%%%%%%%%%%%%%%%%%%%%%%%%%%%%%%%%%%%%%%%%%%%%%%%%%%%%%%%%%%%%%%%%%%%%%%%%%%%%%%%%%%%%%%%%%%%%%%%%%%%%%%%%
%%%%%%%%%%%%%%%%%%%%%%%%%%%%%%%%%%%%%%%%%%%%%%%%%%%%%%%%%%%%%%%%%%%%%%%%%%%%%%%%%%%%%%%%%%%%%%%%%%%%%%%%%%%%%%%%
\section*{Sesión Práctica 4}
\addcontentsline{toc}{section}{Sesión Práctica 4} 


\begin{enumerate}% [label=\sc{Paso} \arabic*.]

\item El profesor te explicará:
	\begin{itemize}
	\item qué es un TDA Lineal,
	\item los 3 tipos de estructuras de datos que se usan para su implementación en Python y
	\item presta especial atención al concepto de nodo como estructura de datos y  cómo se pueden usar para construir estructuras enlazadas lineales. Memoriza los pasos que hay que realizar para las operaciones básicas en estructuras simplemente enlazadas (pág. \pageref{subsec:operacionesBasicasSimplemente}).
	
	\textbf{Recuerda:} No es lo mismo el concepto de nodo como elemento matemático (ver árboles y grafos) que el concepto de nodo como estructura de datos.
	\end{itemize}

\item Pon en práctica el uso de estructuras enlazadas con índices. Realiza el ejercicio \ref{sec:listasimplementeenlazada} considerando dos casos: que la lista enlazada tenga cabecera y que la lista enlazada no tenga cabecera.

Lo ideal es que cada miembro del grupo desarrolle uno de los casos y después lo explique y  ponga a disposición del otro miembro del grupo.

\end{enumerate}




\


\


%%%%%%%%%%%%%%%%%%%%%%%%%%%%%%%%%%%%%%%%%%%%%%%%%%%%%%%%%%%%%%%%%%%%%%%%%%%%%%%%%%%%%%%%%%%%%%%%%%%%%%%%%%%%%%%%
%%%%%%%%%%%%%%%%%%%%%%%%%%%%%%%%%%%%%%%%%%%%%%%%%%%%%%%%%%%%%%%%%%%%%%%%%%%%%%%%%%%%%%%%%%%%%%%%%%%%%%%%%%%%%%%%
\section*{Sesión Práctica 5}
\addcontentsline{toc}{section}{Sesión Práctica 5} 


\begin{enumerate}% [label=\sc{Paso} \arabic*.]

\item El profesor te recordará algunas técnicas de búsqueda  y algunos algoritmos de ordenación que ya viste en el curso anterior (\cm[black]{sort\_bubble()}, \cm[black]{sort\_selection()}, \cm[black]{sort\_insertion()}). También te explicará nuevos métodos (\cm[black]{sort\_busket()} y \cm[black]{sort\_merge()}.

\item Construye un módulo de búsqueda. Ejercicio \ref{sec:moduloBusqueda}.


\item Construye también un  módulo de ordenación. Ejercicio \ref{sec:moduloOrdenacion}.




\item Pon en práctica el uso de estructuras enlazadas con índices. Realiza el ejercicio \ref{sec:listasimplementeenlazada} considerando dos casos: que la lista enlazada tenga cabecera y que la lista enlazada no tenga cabecera.

Es usual que en el examen teórico se platee un ejercicio muy relacionado con esta cuestión.

\end{enumerate}








