% !TEX root = ../EjercPracticas.tex



\begin{definicion}[TDA Árbol Binario]{}  \label{def:arbolBinario}
Un \textbf{árbol binario} es un árbol que \textbf{siempre} tiene dos subárboles que reciben el nombre de hijo (o subárbol) izquierdo e hijo (o subárbol) derecho. En un árbol binario los subárboles pueden ser vacíos.

\noindent Las especificaciones informales del constructor y algunos métodos recursivos son:


\begin{itemize}
\item \cm[black]{TreeBinary() : TreeBinary}. Crea un nuevo árbol binario.

\item \cm[black]{mostrar(tipo="tipo") : None}.  Muestra el contenido del árbol según el tipo de recorrido: prefijo, infijo, postfijo.
	\begin{itemize}
	\item[$\bullet$] Si tipo="\/infijo".  Mostrar los elementos de un árbol como si fuera una expresión aritmética.
	\item[$\bullet$] Si tipo="prefijo".  Mostrar los elementos de un árbol como si fuera una estructura de directorio.
	\end{itemize}

\item Algunos de los métodos indicados en la definición de Árbol se pueden optimizar para este tipo de árboles.
\end{itemize}

\end{definicion}

