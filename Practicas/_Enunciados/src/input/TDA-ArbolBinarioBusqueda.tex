% !TEX root = ../EjercPracticas.tex

\begin{definicion}[TDA Árbol Binario de Búsqueda]{}  \label{def:arbolBinarioBusqueda}
Un \textbf{árbol binario de búsqueda} es un árbol binario que cumple la siguientes 2 propiedades:
\begin{itemize}
\item El valor del hijo izquierdo es menor que el valor de la raíz.
\item El valor del hijo derecho es mayor que el valor de la raíz.
\end{itemize}

Por recursividad, se cumplirá que
\begin{itemize}
\item todos los descendiente de la rama izquierda son menores que el valor de la raíz.
\item todos los descendiente de la rama derecha son mayores que el valor de la raíz.
\end{itemize}


\noindent Las especificaciones de algunos métodos propios de este TDA son:


\begin{itemize}
\item \cm[black]{TreeBinarySearch() : TreeBinarySearch}. Crea un nuevo árbol binario de búsqueda.

\item \cm[black]{search(valor) : bool}. Indica si el valor se encuentra en el árbol.

\item \cm[black]{add(valor) : None}. Añadir un elemento en el árbol. Los nuevos elemento se añaden siempre como nodos hojas. 

\item \cm[black]{min() : valor}. Busca el valor mínimo de un árbol.

\item \cm[black]{max() : valor}. Busca el valor máximo de un árbol.

\item \cm[black]{remove(valor) : None}. Elimina el nodo que tiene un valor. El árbol resultante tiene que seguir siendo un árbol binario de búsqueda.
\end{itemize}

\

Mira en la teoría cómo se deben implementar cada uno de los métodos.

\end{definicion}

